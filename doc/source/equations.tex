%--*-coding:utf-8-unix-*--
\documentclass[a4j,11pt]{jarticle}
\usepackage{txfonts}

\setlength{\oddsidemargin}{0truemm}
\setlength{\hoffset}{-0.4truemm}
\setlength{\voffset}{-0.4truemm}
\setlength{\textheight}{247truemm}
\setlength{\textwidth}{160truemm}
\setlength{\headheight}{0truemm}
\setlength{\topmargin}{0truemm}
\setlength{\headsep}{0truemm}

\pagestyle{empty}

\usepackage{graphicx}% Include figure files

\begin{document}
\renewcommand{\baselinestretch}{1.12}\small\normalsize
\begin{center}
  \textbf{\Large 有限体積法を高速化するための 領域特化言語のC++への埋め込み}
  
  \textbf{\large Design of an embedded domain-specific language for\\ 
accelerating finite volume method with Boost.Proto}
\end{center}
\begin{center}  
  \textbf{伊藤 正勝}

  \textbf{Masakatsu Ito}

\end{center}

演算子\(  d^2 / dx^2 - s\ \hat{Id} \)を離散化して行列で表現するためのプログラミング

\[ k \ \frac{d}{dx} \cdot  \frac{d}{dx} - s\ \hat{Id}  \]


一次元熱伝導問題の解き方

\[ k \frac{d^2}{dx^2} T(x) - s \, ( T(x) - T_\infty ) = 0    \]

をグリッド内の検査体積 \(\Delta x\)ごとに

\[ \int_{\Delta x_i} \left(  k \frac{d^2}{dx^2} T(x) - s \, T(x)   \right) dx =  - \int_{\Delta x_i} s \, T_\infty dx  \]

のように離散化して、連立方程式

\[ \frac{k}{\Delta x} T_{i-1} - \frac{2 k}{\Delta x}T_i +  \frac{k}{\Delta x} T_{i+1} + s  \Delta x\ T_i =  - s\ T_\infty \Delta x   \]


\begin{eqnarray}
\frac{k}{\Delta x} T_{i-1} - \frac{2 k}{\Delta x}T_i +  \frac{k}{\Delta x} T_{i+1}    \\
 +  s  \Delta x\ T_i  =  - s\ T_\infty \Delta x  \\
\end{eqnarray}


を得る。


境界条件を満たすように、連立方程式を修正して、反復法で解く。



拡散方程式

グリッド内の検査体積 \(\Delta V\)(、その境界面\(\Delta A\))ごとに積分


%\[ \int_{\Delta A} {\bf \nabla} \cdot ( \Gamma \,  {\bf \nabla} \phi )\, dA + \int_{\Delta V} S_\phi \,dV  = 0 \]
\[ \int_{\Delta V} {\bf \nabla} \cdot ( k \,  {\bf \nabla} T )\, dV + \int_{\Delta V} S_T \,dV  = 0 \]

\begin{eqnarray}
 \int_{\Delta V} {\bf \nabla} \cdot ( k \,  {\bf \nabla} T )\, dV & \\
+  \int_{\Delta V} S_T \,dV &  = 0  \\ 
\end{eqnarray}

 
\medskip

\[ C \,{\bf t} = {\bf b} \]


有限体積法テンプレートで解く場合

    グリッドオブジェクト

        検査体積 \( \Delta x_i\) 、 境界条件 \( T(0) = T_B , d T(L) / dx = 0 \)

    グリッドオブジェクトが離散化を行う。

        演算子 \(  k\ d^2 / dx^2 - s\ \hat{Id} \) → 行列 \(C\)

        定数項 \( - s \, T_\infty \) → ベクトル \( \bf b \)

            意味論モデルで数式テンプレートを行列、ベクトルに変換

    反復解法ソルバで、行列方程式 \( C {\bf t} = {\bf b} \) を解いて、温度分布 \( \bf t\) を得る。
        意味論モデルで行列、ベクトルの数式を、配列要素計算コードに変換



演算子から行列へ
 

\[ k \frac{ d^2 }{ dx^2} \rightarrow  k \left( \begin{array}{@{\,}ccccc@{\,}} \ddots & 1 / \Delta x & & & \\ 1 / \Delta x & - 2 / \Delta x & 1 / \Delta x & & \\ & 1 / \Delta x & - 2 / \Delta x & 1 / \Delta x & \\ & & 1 / \Delta x & - 2 / \Delta x & 1 / \Delta x \\ & & & 1 / \Delta x & \ddots \end{array} \right)  \]


\[  \frac{ d^2 }{ dx^2} \rightarrow  \left( \begin{array}{@{\,}ccccc@{\,}} \ddots & 1 / \Delta x & & & \\ 1 / \Delta x & - 2 / \Delta x & 1 / \Delta x & & \\ & 1 / \Delta x & - 2 / \Delta x & 1 / \Delta x & \\ & & 1 / \Delta x & - 2 / \Delta x & 1 / \Delta x \\ & & & 1 / \Delta x & \ddots \end{array} \right)  \]


微分方程式の演算子部分
 \[  k\ \frac{d^2}{dx^2} - s\ \hat{Id} \]
 


移流拡散問題

\[ {\bf \nabla} \cdot ( \rho {\bf u} \phi) = {\bf \nabla} \cdot ( \Gamma \, {\bf \nabla} \phi )  + S_\phi \]

をグリッド内の検査体積 \(\Delta V\)、その境界面\(\Delta A\)ごとに

\[ \int_{\Delta A} {\bf n} \cdot ( \rho {\bf u} \phi )\, dA = \int_{\Delta A} {\bf n} \cdot ( \Gamma \,  {\bf \nabla} \phi )\, dA + \int_{\Delta V} S_\phi \,dV  \]
 




\end{document}
